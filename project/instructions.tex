\documentclass[11pt,a4paper]{article}
\usepackage{amsthm, amsmath, amsfonts, exscale, latexsym, graphicx, setspace, tabularx, booktabs, microtype}
\usepackage{times, pdfpages, wasysym}
\usepackage{enumitem}
%\usepackage[mtpcal,mtphrb,subscriptcorrection]{mtpro2}
\usepackage{hyperref}
\usepackage[ngerman]{babel}
\def\tomonth{\ifcase\month\or January\or February\or March\or April\or May\or June\or July\or August\or September\or October\or November\or December\fi\space\number\year}
\textwidth162mm
\textheight23.5cm
\topmargin-0.75cm
\headheight0cm
\oddsidemargin-1.4mm \evensidemargin-1.4mm \unitlength1cm

\onehalfspacing
\begin{document}


\begin{center}
\bfseries\huge Instructions for the Programming Project
\end{center}
The aim of the project is to download and visualize/summarize some data using Python, Pandas, and Matplotlib, in teams of two. You are free to choose the data that you want to work with; some ideas are CoViD as used in class, financial data from Yahoo Finance (there are packages for interfacing Python with it), CO2 and climate change, macroeconomics, etc.

The end result should be a nice report containing some plots, maybe some other analysis, and a verbal description of the results. The report can be written using Word or \LaTeX\ (this file uses the latter), or as a Jupyter notebook. If you don't use Jupyter to produce your final report, then you should include the code in your (Word or \LaTeX) document, and also provide it separately in runnable form, i.e., as a \textbf{.py} file or notebook. \textbf{Important}: Any code that you send me should run without errors! In Jupyter, you can make sure it does by doing \texttt{Restart and run all} before submitting.

There is no explicit page or word limit, just be reasonable. The deadline for handing in the report is May 31st, 11:59pm. Submission is either by email to \href{mailto:simon.broda@hslu.ch}{simon.broda@hslu.ch}, or via Ilias. Please use the latter if your submission is large (more than 1MB). Several files should be combined into a zip archive for submission on Ilias.

As for grading, here are some of the things that I will reward:
\begin{itemize}
\item If you choose to use a Jupyter notebook: using MarkDown to format text.
\item If you choose to use a Jupyter notebook: having some of the code in functions inside a module (\texttt{.py} file) and importing them into the notebook.
%\item Collaborating using git and hosting the analysis on Github.
\item Combining data from different sources.
\item Transforming the data in some way (e.g., moving averages).
\item Pretty graphs (labels, titles, line styles, colors etc.) Don't forget to mention the data sources if not obvious from the code.
\item In general, a well-written report with an interesting analysis.
\end{itemize}
This does \emph{not} mean that you have to do all of the above.


The last section of the report should be a reflection on your work. Which problems did you encounter and how did you solve them? What was easy, what was hard? How would you rate the end result?\\\\
\noindent
\textbf{Best of luck}!
\end{document} 